\documentclass[a4paper]{article}

\usepackage{color}
\usepackage[top=1in,bottom=1in,left=1.2in,right=1.2in]{geometry}
\usepackage{hyperref}
\usepackage[small]{titlesec}

\usepackage{graphicx}
\usepackage{color}

\usepackage[]{minted}
\usepackage{amsmath}
\newenvironment{code}{\captionsetup{type=listing}}


\newcommand{\todo}[1]{\textcolor{red}{\textbf{TODO:} #1}}

\title{Assignment 1 \\ \begin{small}\url{FULL_GITHUB_REPOSITORY_URL}\end{small}}
\author{Group 22 \and Matthew Evans \\ mre042000 \and Ruochen Meng \\ rxm220120 \and Saul Garay \\ SAG180009}

%\author{Full Name \\ Net ID \and Full Name \\ Net ID}

\date{}

\begin{document}
\maketitle
% \input{begin.tex}


\section{Implementation Details}

\begin{listing}[ht]
  \begin{minted}[frame=single,framesep=10pt]{python}
  import numpy  
  print("this is a piece of code")
\end{minted}
  \caption{Example of a Code Piece.}
  \label{lst:eg}
\end{listing}

\subsection{Unigram and bigram probability computation}

\subsection{Smoothing}

\subsection{Unknown word handling}

\begin{enumerate}
  \item Sort tokens in descending order by occurrence count.
  \item Iterate over token counts, accumulating mass, \(M\), i.e.,
        \[
          \text{M}_{\text{cum}} = \sum_{t_i}^{t_n}\text{C}(t_i)
        \]
        If \((\text{M}_{\text{cum}} / M_{\text{total}}) < \text{M}_{\text{coverage}}\), assign the token's mass to the \texttt{<UNK>} token.
\end{enumerate}

\subsection{Implementation of perplexity}

\section{Eval, Analysis and Findings}

\section{Others}



\end{document}
